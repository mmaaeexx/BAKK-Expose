\chapter{Problemstellung}
\label{ch:Problemstellung}
Die Kommunikation heutzutage ist stark von den unterschiedlichsten Computernetzen abhängig. Diese Computernetze ermöglichen es Menschen und Geräten zu kommunizieren, egal wo sie sich befinden.
\\
Grundsätzlich werden diese Netze in \glqq Lokale Netze(LAN)\grqq{} und \glqq Nichtlokale Netze(WAN)\grqq{} unterschieden. Während wir in der täglichen Kommunikation via Messenger oder E-Mail hauptsächlich über \wan-Netze kommunizieren erfolgt die Kommunikation innerhalb von Produktionsanlagen heute nach wie vor hauptsächlich im \lan. Im Zuge der Arbeit berücksichtigen wir die unterschiedlichsten Typen vom Kommunikationsnetzen, relevant ist aber hauptsächlich die Kommunikation im \lan.
\\
Im Zuge eines globalen angelegten Projektes zur Netzwerksegmentierung sind alle \so{} in \ot{}-Netzen damit konfrontiert die Kommunikationsbeziehungen auf ihre Systeme zu identifizieren und entsprechend auf ihre Notwendigkeit zu bewerten. Durch Anreicherung der identifizierten Daten mit zusätzlichen Datenquellen wie ITSM, DNAC, Firewall – Logs soll es dem AO ermöglicht werden die Kommunikationsbeziehungen seiner Systeme zu identifizieren.
Durch die Verknüpfung der erfassten Daten aller AO soll eine “Application Landscape” für den kompletten Konzern dargestellt werden. 
